%===================================== CHAP 1 =================================

\chapter{Introduction}

Here i should write about why i am researching this.

\section{Motivation}


\section{Research Goals}

%vurdere å ha en overall goal?

\paragraph{1 - Set up a pipeline}
The first goal of this study will be to setup a software pipeline in the most beneficial way. The pipeline should extract examples from Wikipedia. The output of the pipeline should be a searchable database of the examples.

\paragraph{2 - Define a good example by using their structure and content}
To be able to create a high quality database consisting of examples, a deeper understanding is needed. Therefor the second research goal in this study will focus on the nature of the examples. The study will look into how the examples are structured, what properties are preferable in certain circumstances and what the content of an examples should be in regards to its domain. 

\paragraph{3 - Create and populate a database of examples}
\begin{itemize}
    \item To be able to populate the database, we first have to create the database. The type of the database is essential, since it will directly affect the process of inserting and retrieving examples. The modeling of the database in terms of possible entities, relations and properties will also have to be considered.
    \item When the database is created the next step will be to populate it with examples. Based on the models decided for the database, queries will be created that inserts all relevant examples.
\end{itemize}


\paragraph{4 - Search examples }
//Tried to rewrite first point based on feedback, but i'm not satisfied
\begin{itemize}
    \item To make use of the database, a set of queries will have to be defined. We have to decide what information we need to present for the user. Then we can define the queries needed to fetch this information from the database.
    \item To make use of the database, a set of queries will have to be defined. The queries returns results, which the interface will make available to the user. What of the results content to make available, has to be decided.
    \item An interface is needed to make the user able to execute the queries. The interface also has to present the result of the queries to the user.
\end{itemize}


\section{Structure of thesis}
This paper will reflect the process of creating a searchable database consisting of examples. In order to accomplish the end results, four research goals has been formulated, which this paper will attempt to achieve. The project starts with a large dump of source data from Wikipedia, for this data to be use full, it first has to be filtered and structured by using a pipeline (Research Goal 1). To better understand examples, and thus improving search results, the paper will also look into what defines a good example (Research Goal 2). The end result of the pipeline is a database populated with examples. The database has to store the examples in a way that makes them easy to retrieve (Research Goal 3). Finally, a user interface is needed to search for examples from the database. The user interface will have queries for retrieving relevant examples and an search interface for the user to query the system by keywords. 

Chapter two will explore the field of information retrieval and extraction, for similar work as in this paper. In particular, it will look for data extraction from wiki sites, and data retrieval of semi-structured text. In addition, the chapter will also explore an alternative approach, SMILA, and look at some techniques used in this project.

Chapter three will elaborate on the conceptual design of the project. The main reasons for using a pipeline and how the pipeline refines the source data, will be explained. The chapter will also explain how the project intend to search for examples. Finally a thorough analysis of examples, reflecting Research Goal 2, will be presented.

Chapter four explains the implementation of the system. The chapter will start by stepping through the pipeline, explaining each sub process used to refine the data. Meanwhile it will elaborate on the role of different files in the project, what libraries used to help and storage systems used. The search interface created for fetching the examples will be described, including its interaction towards rest of the system. Finally a more detailed explanation will be given for the main tools used by the system.

Chapter five will answer to how the Research Goals were accomplished. It will try to answer for each one of them in order, either by elaborating in how the goal is achieved already by the system or the paper, or by conducting experiments to verify that the system produces the intended results.

Chapter six...


\cleardoublepage