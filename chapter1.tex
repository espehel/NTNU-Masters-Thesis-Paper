%===================================== CHAP 1 =================================

\chapter{Introduction} \label{cap_1}

\section{Motivation}

Learning has always been, and will most likely always be one of the central pillars of human success. Humans are very passionate about learning, and therefore continually improve on techniques for acquiring knowledge. The traditional way of learning has been \textit{Passive Learning}, but \textit{Active Learning} has been more prevailing, especially for schools and universities \cite{active-learning}. Although the classical passive learning technique, lectures, are still heavily used, discussions, problem solving, tasks and student presentations have become a big part of a normal school day. J. P. Lalley and R. H. Miller discusses the learning pyramid produced by Edgar Dale \cite{learning_pyramid}. Figure \ref{fig:lt} depicts the learning pyramid displaying different learning techniques, and how much knowledge you retain after using them. Edgar Dale's research has been disputed since the original data, regarding retaining knowledge, was not recorded. Despite the retention values may be inaccurate, it gives an overview of different learning techniques. 

\begin{figure}[h] 
\caption{A typical representation of the learning pyramid, showing passive and active learning techniques and how much knowledge retained.}
\includegraphics[width=\textwidth]{learning_triangle}
\label{fig:lt}
\end{figure}

Almost at the bottom of the pyramid, the technique \textit{Practice by doing} can be found. Practice by doing is a very popular technique, often in the form of assignments based on solving tasks or problems. From personal experience in solving tasks, making use of examples that solves similar tasks can be of great help. Finding helpful examples among the massive information accessible through the web is sometimes challenging though.

Educational Technology could be applied to improve the process of finding helpful examples. The article \textit{Facilitating Learning} \cite{e-learning} uses AECT's\footnote{The Association for Educational Communications and Technology} definition of Educational Technology, "Educational technology is the study and ethical practice of facilitating learning and improving performance by creating, using, and managing appropriate technological processes and resources." The article highlights the use of the term \textit{facilitation} in the definition. They argue that Educational technology's primary purpose is to \textbf{help} people learn, not control or manage it, which older definitions leaned towards. This project will facilitate the users' learning by creating a technological process that will gather and manage a collection of examples. The collection can be used as a tool for learning new subjects or solving tasks.

To accomplish this, the thesis will look closer into the examples themselves. We want to discover what separates a good example from a bad one, and if there is a particular structure reflecting the quality of the example. Examples are written by people, learning how the author of an example creates it, is also features of examples the thesis will attempt to examine. With this knowledge, we will explore the possibility of creating a database consisting purely of examples. Publicly available examples found in Wikipedia will be transformed into structured example objects, and then inserted into a database that users can access through a search interface. 



\section{Research Goals}

The overall goal of our work is to build a platform where users can find examples, which will aid them in learning. To help aim the work towards the thesis' overall goal, four research questions have been defined. Completion of these goals will ensure good quality of the implemented system.


\paragraph{1 - Set up a pipeline}
The first goal of this study will be to setup a software pipeline in the most beneficial way. The pipeline should extract examples from Wikipedia. The output of the pipeline should be a searchable database of the examples.

\paragraph{2 - Define a good example by using their structure and content}
To be able to create a high quality database consisting of examples, a deeper understanding of the examples themselves is needed. Therefore the second research goal in this study will focus on the nature of the examples. The study will look into how the examples are structured, what properties are preferable in certain circumstances and what the content of an example should be in regards to its domain. 

\paragraph{3 - Create and populate a database of examples}
\begin{itemize}
    \item To be able to populate the database, we first have to create it. Which database management system to use is essential, since it will directly affect the process of inserting and retrieving examples. The modeling of the database in terms of possible entities, relations and properties will also have to be considered.
    \item When the database is created the next step will be to populate it with examples. Based on the models decided for the database, queries will be created that insert all relevant examples.
\end{itemize}


\paragraph{4 - Implement a user interface for searching examples }
\begin{itemize}
    \item Showing how the created database of examples can be used to be queried and present the relevant example information to the user. This includes different search queries facilitating different objectives that can be served by the example database.
    \item An interface is needed to make the user able to execute the queries. The interface also has to present the result of the queries to the user.
\end{itemize}


\section{Structure of thesis}
This thesis will describe the process of creating a searchable database consisting of examples. In order to accomplish the end results, four research goals have been formulated. The project starts with a large dump of source data from Wikipedia, for this data to be fully used, it first has to be filtered and structured by using a pipeline (Research Goal 1). To better understand examples, and thus improving search results, we will also look into what defines a good example (Research Goal 2). The end result of the pipeline is a database populated with examples. The database has to store the examples in a way that makes them easy to retrieve (Research Goal 3). Finally, a user interface is needed to search for examples from the database. Queries with keywords as user input will be used to retrieve relevant examples.

Chapter 2 will explore the related work in field of information retrieval and extraction. In particular, it will look for data extraction from wiki sites, and data retrieval of semi-structured text. In addition, the chapter will also explore an alternative approach, SMILA, and look at some techniques used in this project.

Chapter 3 will elaborate on the conceptual design of the project. The main reasons for using a pipeline and how the pipeline refines the source data, will be explained. The chapter will also explain how the project intend to search for examples. Finally a thorough analysis of examples, reflecting Research Goal 2, will be presented.

Chapter 4 explains the implementation of the system. The chapter will start by stepping through the pipeline, explaining each sub process used to refine the data. Meanwhile it will elaborate on the role of different files in the project, what libraries used to help and storage systems used. The search interface created for fetching the examples will be described, including its interaction towards rest of the system. Finally a more detailed explanation will be given for the main tools used by the system.

Chapter 5 will discusses whether and how the Research Goals are accomplished. It will try to answer for each one of them in order, either by elaborating in how the goal is achieved already by the system or in the thesis, or by conducting experiments to verify that the system produces the intended results.

Chapter 6 will conclude our thesis. With the results of the experiments in Chapter 5 and the following discussion, the chapter will try to answer whether the research goals have been reached or not. As an ending for the chapter and the thesis, the future of the system will be discussed.


\cleardoublepage