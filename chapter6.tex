%===================================== CHAP 6 =================================

\chapter{Conclusion and Future Work}


\section{Conclusion}
The aim of this project has been to augment the use of examples for learning, by making use of educational technology. A system parsing Wikipedia articles for the extraction of sections containing examples has been created. Four research goals were established in chapter \ref{cap_1} in order to manage the projects work flow into desired results. In chapter \ref{cap_2}, other peoples work regarding text data mining, semi-structured text and Wikimedia was examined, to help discover the usefulness and possibilities of this project. The concept of a pipeline turning raw source data from Wikipedia into an index containing examples, were explained in chapter \ref{cap_3}. In addition, how to search the index was also expressed. To optimize how the system handles the collecting and serving of examples, an analysis of example's structure and content were performed as well. Chapter \ref{cap_4} explained in detail how the defined concept were implemented into a working system. Finally chapter \ref{cap_5} examined the accomplishment of the research goals. Research goal 1, 2 and 3 were summed up and concluded, while two experiments were conducted to evaluate the fourth research goal. The first experiment tested the precision for several keywords used as a search phrase. The second tested the precision of the results when the four different white lists were applied one at a time.

\section{Future Work}
There are several aspects of this project which would benefit from an extended amount of work or research

Firstly the system itself, can be greatly improved by being able to accept examples from different sources. The foundation for doing so already exists in the system, since the idea of extracting examples from different sources have existed since the beginning. For simplicity and saving time, this project has focused on only using Wikipedia as source, and therefor is excessively tailored for Wikipedia articles. But since the different processes in the pipeline is extremely independent, replacing them to accommodate other sources should be a trivial task. Accommodating more sources would result in a richer database of examples, which in turn would help the end user.

Another aspect that could benefit the system is a deeper knowledge of examples, and how they relate to each other. A better understanding could improve both the rating of their relevance score when searched for and displaying related examples. Improvement on finding related examples, can give a natural learning progress when browsing from one example and to its related ones.

The search itself could also be more optimized. Since Elasticsearch was chosen to manage the database of examples, the search API served by the Elasticsearch process could be explored further. Elasticsearch offers a great amount of different customizations that can be applied to the queries used for the search, which would make the search more complex, but also could improve the search results.

%%kategorisere eksempler bedre. Bruke kategorier som taggs istedet. Slik at de bedre definerer eksempler og også da bedrer søking for relaterte

\cleardoublepage